\documentclass{beamer}

\usepackage[slovene]{babel}
\usepackage{amsfonts,amssymb}
\usepackage[utf8]{inputenc}
\usepackage{lmodern}
\usepackage[T1]{fontenc}

\usetheme{Warsaw}

\def\N{\mathbb{N}} % mnozica naravnih stevil
\def\Z{\mathbb{Z}} % mnozica celih stevil
\def\Q{\mathbb{Q}} % mnozica racionalnih stevil
\def\R{\mathbb{R}} % mnozica realnih stevil
\def\C{\mathbb{C}} % mnozica kompleksnih stevil


\def\qed{$\hfill\Box$}   % konec dokaza
\newtheorem{izrek}{Izrek}
\newtheorem{trditev}{Trditev}
\newtheorem{posledica}{Posledica}
\newtheorem{lema}{Lema}
\newtheorem{definicija}{Definicija}
\newtheorem{pripomba}{Pripomba}
\newtheorem{primer}{Primer}
\newtheorem{zgled}{Zgled}
\newtheorem{zgledi}{Zgledi uporabe}
\newtheorem{zglediaf}{Zgledi aritmetičnih funkcij}
\newtheorem{oznaka}{Oznaka}
\newtheorem{dokaz}{Dokaz}

\title{Topološke lastnosti grup}
\author{Gašper Rotar}
\institute{Fakulteta za matematiko in fiziko}




\begin{document}


%%%%%%%%%%%%%%%%%%%%%%%%%%%%%%%%%%%%%%%%%%%%%%%%%%%%%%%%%%%%%%%%%%%%%

\begin{frame}
\titlepage
\end{frame}

%%%%%%%%%%%%%%%%%%%%%%%%%%%%%%%%%%%%%%%%%%%%%%%%%%%%%%%%%%%%%%%%%%%%%

\begin{frame}
\frametitle{Dve preprosti trditvi}

\begin{definicija}
    Naj bo $N$ podgrupa grupe $G$. $N$ je podgrupa \emph{edinka} grupe $G$, označimo $N \triangleleft G$, če za vse $a \in G$ in $n \in N$ velja $ana^{-1} \in N$.
\end{definicija}

\pause

\begin{trditev}
    Naj bo $G$ grupa in $\Delta(G) = \{(g,g) \mid g \in G \} \subseteq G \times G$ njena diagonala.
    Grupa $G$ je komutativna natanko takrat, ko je $\Delta(G)$ podgrupa edinka grupe $G$, $\Delta(G) \triangleleft G$.
\end{trditev}

\end{frame}

\begin{frame}
\frametitle{Dve preprosti trditvi}

\begin{dokaz}
    \begin{itemize}
        \item[($\Rightarrow$)]
            Ker je $G$ komutativna je seveda tudi $G \times G$ komutativna.
            \pause
            $a \cdot b = (\alpha, \alpha) \cdot (\beta, \beta) = (\alpha \cdot \beta, \alpha \cdot \beta) \in \Delta(G)$
            \pause
            $(\alpha, \alpha) \cdot (\alpha^{-1}, \alpha^{-1}) = (\alpha \cdot \alpha^{-1}, \alpha \cdot \alpha^{-1}) = (1,1)$
            \pause
            \bigskip
        \item[($\Leftarrow$)]
            Naj bodo $(\alpha,1), (\alpha^{-1}, 1), (\beta, \beta) \in G \times G$
            \newline
            \pause
            Potem: $(\alpha,1) \cdot (\beta, \beta) \cdot (\alpha^{-1}, 1) = (\alpha\beta\alpha^{-1}, \beta).$
            \newline
            \pause
            Velja: $(\alpha\beta\alpha^{-1}, \beta) \in \Delta$
            \newline
            \pause
            Torej:$\alpha\beta\alpha^{-1} = \beta \Rightarrow \alpha\beta = \beta\alpha$
    \end{itemize}
\end{dokaz}

\end{frame}

\begin{frame}
\frametitle{Dve preprosti trditvi}

\begin{definicija}
    Topološki prostor $X$ je \emph{Hausdorffov}, če za vsaki različni točki $x_1, x_2 \in X$ obstajata odprti okolici $U_1$ in $U_2$ za točki $x_1$ in $x_2$, da $U_1 \cap U_2 = \emptyset$.
\end{definicija}
\pause

\begin{trditev}
    Naslednje izjave so ekvivalentne:
    \pause
    \begin{itemize}
        \item[\rm (i)] Prostor $X$ je Hausdorffov.
        \pause 
        \item[\rm (ii)] Za poljuben $x \in X$ je $\bigcap_{U \in \mathcal{U}} \overline{U} = \{x\}$, kjer je $\mathcal{U}$ družina vseh okolic $x$.
        \pause 
        \item[\rm (iii)] Diagonala $\Delta(X) = \{(x,x) \mid x \in X \} $ je zaprt podprostor produkta $X \times X$
    \end{itemize}
\end{trditev}

\end{frame}

\begin{frame}
\frametitle{Dve preprosti trditvi}

\begin{dokaz}
    Tu pride dokaz haudorf
\end{dokaz}

\end{frame}


%%%%%%%%%%%%%%%%%%%%%%%%%%%%%%%%%%%%%%%%%%%%%%%%%%%%

\begin{frame}
\frametitle{Konjugiranostni razredi}

\begin{definicija}
    Element $y$ grupe $G$ je \emph{konjugiran} elementu $x$ iz $G$, če obstaja tak $g \in G$, da je $y = gxg^{-1}$.
\end{definicija}
\bigskip
\pause

\begin{trditev}
    Konjugiranost je ekvivalenčna relacija.
\end{trditev}

\end{frame}


\begin{frame}
\frametitle{Konjugiranostni razredi}

\begin{dokaz}
    tu pride dokaz Konjugiranostni
\end{dokaz}

\end{frame}

\begin{frame}
\frametitle{Konjugiranostni razredi}

\begin{zgled}
    Če je $G$ komutativna potem za vsak $x,g \in G$ velja:
    \newline
    \[gxg^{-1} = xgg^{-1} = x\]
    \pause
\end{zgled}

\begin{zgled}
    Kvaternionska grupa $Q = \{\pm 1, \pm i, \pm j, \pm k\}$ ni komutativna
    \newline
    \pause
    Konjugiranostni razredi so $\{1\}, \{-1\}, \{\pm i\}, \{\pm j\}, \{\pm k\}$
    \pause
\end{zgled}

\begin{zgled}
    Podobne matrike: $B = P^{-1}AP$
\end{zgled}


\end{frame}













\end{document}