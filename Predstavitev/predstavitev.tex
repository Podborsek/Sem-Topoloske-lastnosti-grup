\documentclass{beamer}

\usepackage[slovene]{babel}
\usepackage{amsfonts,amssymb}
\usepackage[utf8]{inputenc}
\usepackage{lmodern}
\usepackage[T1]{fontenc}

\usetheme{Warsaw}

\def\N{\mathbb{N}} % mnozica naravnih stevil
\def\Z{\mathbb{Z}} % mnozica celih stevil
\def\Q{\mathbb{Q}} % mnozica racionalnih stevil
\def\R{\mathbb{R}} % mnozica realnih stevil
\def\C{\mathbb{C}} % mnozica kompleksnih stevil


\def\qed{$\hfill\Box$}   % konec dokaza
\newtheorem{izrek}{Izrek}
\newtheorem{trditev}{Trditev}
\newtheorem{posledica}{Posledica}
\newtheorem{lema}{Lema}
\newtheorem{definicija}{Definicija}
\newtheorem{pripomba}{Pripomba}
\newtheorem{primer}{Primer}
\newtheorem{zgled}{Zgled}
\newtheorem{zgledi}{Zgledi uporabe}
\newtheorem{zglediaf}{Zgledi aritmetičnih funkcij}
\newtheorem{oznaka}{Oznaka}
\newtheorem{dokaz}{Dokaz}

\title{Topološke lastnosti grup}
\author{Gašper Rotar}
\institute{Fakulteta za matematiko in fiziko}




\begin{document}


%%%%%%%%%%%%%%%%%%%%%%%%%%%%%%%%%%%%%%%%%%%%%%%%%%%%%%%%%%%%%%%%%%%%%

\begin{frame}
\titlepage
\end{frame}

%%%%%%%%%%%%%%%%%%%%%%%%%%%%%%%%%%%%%%%%%%%%%%%%%%%%%%%%%%%%%%%%%%%%%

\begin{frame}
\frametitle{Dve preprosti trditvi}

\begin{definicija}
    Naj bo $N$ podgrupa grupe $G$. $N$ je podgrupa \emph{edinka} grupe $G$, označimo $N \triangleleft G$, če za vse $a \in G$ in $n \in N$ velja $ana^{-1} \in N$.
\end{definicija}

\pause

\begin{trditev}
    Naj bo $G$ grupa in $\Delta(G) = \{(g,g) \mid g \in G \} \subseteq G \times G$ njena diagonala.
    Grupa $G$ je komutativna natanko takrat, ko je $\Delta(G)$ podgrupa edinka grupe $G$, $\Delta(G) \triangleleft G$.
\end{trditev}

\end{frame}

\begin{frame}
\frametitle{Dve preprosti trditvi}

\begin{dokaz}
    \begin{itemize}
        \item[($\Rightarrow$)]
            Ker je $G$ komutativna je seveda tudi $G \times G$ komutativna.
            \pause
            $a \cdot b = (\alpha, \alpha) \cdot (\beta, \beta) = (\alpha \cdot \beta, \alpha \cdot \beta) \in \Delta(G)$
            \pause
            $(\alpha, \alpha) \cdot (\alpha^{-1}, \alpha^{-1}) = (\alpha \cdot \alpha^{-1}, \alpha \cdot \alpha^{-1}) = (1,1)$
            \pause
            \bigskip
        \item[($\Leftarrow$)]
            Naj bodo $(\alpha,1), (\alpha^{-1}, 1), (\beta, \beta) \in G \times G$
            \newline
            \pause
            Potem: $(\alpha,1) \cdot (\beta, \beta) \cdot (\alpha^{-1}, 1) = (\alpha\beta\alpha^{-1}, \beta).$
            \newline
            \pause
            Velja: $(\alpha\beta\alpha^{-1}, \beta) \in \Delta$
            \newline
            \pause
            Torej:$\alpha\beta\alpha^{-1} = \beta \Rightarrow \alpha\beta = \beta\alpha$
    \end{itemize}
\end{dokaz}

\end{frame}

\begin{frame}
\frametitle{Dve preprosti trditvi}

\begin{definicija}
    Topološki prostor $X$ je \emph{Hausdorffov}, če za vsaki različni točki $x_1, x_2 \in X$ obstajata odprti okolici $U_1$ in $U_2$ za točki $x_1$ in $x_2$, da $U_1 \cap U_2 = \emptyset$.
\end{definicija}
\pause

\begin{trditev}
    Naslednje izjave so ekvivalentne:
    \pause
    \begin{itemize}
        \item[\rm (i)] Prostor $X$ je Hausdorffov.
        \pause 
        \item[\rm (ii)] Za poljuben $x \in X$ je $\bigcap_{U \in \mathcal{U}} \overline{U} = \{x\}$, kjer je $\mathcal{U}$ družina vseh okolic $x$.
        \pause 
        \item[\rm (iii)] Diagonala $\Delta(X) = \{(x,x) \mid x \in X \} $ je zaprt podprostor produkta $X \times X$
    \end{itemize}
\end{trditev}

\end{frame}

\begin{frame}
\frametitle{Dve preprosti trditvi}

\begin{dokaz}
    (i) $\Rightarrow$ (ii) Naj $X$ Hausdorffov in $y \neq x$.
    \newline
    \pause
    $x \in U, y \in V, U \cap V = \emptyset$
    \newline
    \pause
    Torej $y \notin \overline{U}$
    \bigskip
    \newline
    \pause
    (ii) $\Rightarrow$ (iii) Pokažimo, da $\Delta^C$ odprta v $X \times X$
    \newline
    \pause
    $(x,y) \in \Delta^C$ obstaja $U$, da $x \in U$ in $y \notin \overline{U}$
    \bigskip
    \newline
    \pause
    (iii) $\Rightarrow$ (i) Naj bo $\Delta^C$ odprta
    \newline
    \pause
    Za $x \neq y$ je $(x,y) \in \Delta^C$
    \newline
    \pause
    $(x,y) \in $ $U \times V$
    \pause
    \qed
\end{dokaz}

\end{frame}


%%%%%%%%%%%%%%%%%%%%%%%%%%%%%%%%%%%%%%%%%%%%%%%%%%%%

\begin{frame}
\frametitle{Konjugiranostni razredi}

\begin{definicija}
    Element $y$ grupe $G$ je \emph{konjugiran} elementu $x$ iz $G$, če obstaja tak $g \in G$, da je $y = gxg^{-1}$.
\end{definicija}
\bigskip
\pause

\begin{trditev}
    Konjugiranost je ekvivalenčna relacija.
\end{trditev}

\end{frame}


\begin{frame}
\frametitle{Konjugiranostni razredi}

\begin{dokaz}
    \begin{itemize}
        \item<1->Refleksivnost: $x = exe$
        \bigskip
        \item<2->Simetričnost:  $y = gxg^{-1} \Rightarrow x = g^{-1}yg$
        \bigskip
        \item<3->Tranzitivnost: $y = gxg^{-1}, z = hyh^{-1} \Rightarrow z = hgxg^{-1}h^{-1}$
    \end{itemize}
\end{dokaz}

\end{frame}

\begin{frame}
\frametitle{Konjugiranostni razredi}

\begin{zgled}
    Če je $G$ komutativna potem je vsak element v svojem razredu.
    \medskip
    \pause
    \[ghg^{-1} = gg^{-1} = g\]
    \pause
    \[ghg^{-1} = h \Rightarrow gh = hg\]
    \pause
\end{zgled}

\begin{zgled}
    Kvaternionska grupa $Q = \{\pm 1, \pm i, \pm j, \pm k\}$ ni komutativna
    \newline
    \pause
    Konjugiranostni razredi so $\{1\}, \{-1\}, \{\pm i\}, \{\pm j\}, \{\pm k\}$
    \pause
\end{zgled}

\begin{zgled}
    Podobne matrike: $B = P^{-1}AP$
\end{zgled}

\end{frame}


\begin{frame}
\frametitle{Konjugiranostni razredi}

\begin{izrek}
    Naj bo $G$ grupa in označimo $U_h = \{ghg^{-1} \mid g \in G \}$, to je konjugiranostni razred elementa $h$.
    \pause
    Množica $\Theta = \bigcup_{h \in G}\{U_h\}$ je baza topologije na $G$.
    \pause
\end{izrek}


\begin{pripomba}
    Topologijo, ki jo kot baza podaja $\Theta$ bomo imenovali \emph{konjugiranostna topologija} in jo označili z $\mathcal{T}(G)$.
    \pause
\end{pripomba}

\begin{dokaz}
    \begin{itemize}
        \item $\Theta$ je pokritje
        \pause
        \item$U_h$ in $U_k$ bazni množici
        \newline
        \pause
        $x \in U_h \cap U_k \Rightarrow U_h = U_k$
        \pause
        \qed
    \end{itemize}
\end{dokaz}

\end{frame}

%%%%%%%%%%%%%%%%%%%%%%%%%%%%%%%%%%%%%%%%%%%%%%%%%%%%%%

\begin{frame}
\frametitle{Uporaba topologije $\mathcal{T}(G)$}

\begin{izrek}
    $G$ je komutativna natanko takrat, ko je $\mathcal{T}(G)$ Hausdorffova.
    \pause
\end{izrek}

\begin{dokaz}
    \begin{itemize}
        \item[($\Rightarrow$)] $G$ komutativna $\Rightarrow U_x = \{x\}$
        \newline
        \pause
        $\mathcal{T}(G)$ diskretna
        \bigskip
        \item[($\Leftarrow$)] $\mathcal{T}(G)$ Hausdorffova
        \newline
        \pause
        Recimo da $y \neq x$ in $y \in U_x$
        \newline
        \pause
        $x \in U_h, y \in U_k$ in $U_h \cap U_k = \emptyset$
        \newline
        \pause
        $x \in U_x \cap U_h$ in $y \in U_x \cap U_k$ torej $U_h = U_x = U_h$
        \pause
        $\Rightarrow\!\Leftarrow$
    \end{itemize}
\end{dokaz}

\end{frame}

%\item[($\Rightarrow$)] $G$ je Abelova natanko takrat, ko $gh = hg $, za vsak $h,g \in G$, ekvivalentno $ghg^{-1} = h$, za vsak $g,h \in G$, kar pa ravno pomeni $U_h = \{ h\}$ za vsak $h \in G$.
%To pomeni, da so v $\mathcal{T}(G)$ enojci odprti, torej je to diskretna topologija. Diskretna topologija je seveda Hausdorffova.
%\item[($\Leftarrow$)] Privzemimo sedaj, da je $\mathcal{T}(G)$ Hausdorffov. Imejmo poljuben $x \in G$ in recimo, da obstaja $y \in U_x$ različen od $x$.
%Ker je $G$ Hausdorffova obstajata odprti okolici $U_h$ in $U_k$, da $x \in U_h, y \in U_k$ in $U_h \cap U_k = \emptyset$.
%Toda to pomeni, da $x \in U_x \cap U_h$ in $y \in U_x \cap U_k$ torej $U_h = U_x = U_h$, kar pa je v protislovju z $U_h \cap U_k = \emptyset$.
%To pomeni, da $U_x = \{x\}$, kar pa je ravno ekvivalentno komutativnosti $G$.




\begin{frame}
\frametitle{Uporaba topologije $\mathcal{T}(G)$}

\begin{izrek}
    $H \triangleleft G$, če in samo če je $H$ zaprt v $\mathcal{T}(G)$.
    \pause
\end{izrek}

\begin{dokaz}
    ($\Rightarrow$)
    $H \triangleleft G$, pokažemo da $G - H$ odprta.
    \newline
    \pause
    Naj bo $x \in G - H$ in $U_x$ konjugiranostni razred $x$.
    \newline
    \pause
    $U_x \cap H \neq \emptyset \Rightarrow \exists g \in G, h \in H: h = gxg^{-1}$.
    \newline
    \pause
    Sledi $x = g^{-1}hg$, ker $H$ edinka potem $x \in H$.     
    \pause
    $\Rightarrow\!\Leftarrow$
\end{dokaz}

\end{frame}

\begin{frame}
    \frametitle{Uporaba topologije $\mathcal{T}(G)$}
    
    \begin{izrek}
        $H \triangleleft G$, če in samo če je $H$ zaprt v $\mathcal{T}(G)$.
    \end{izrek}
    
    \begin{dokaz}
        ($\Leftarrow$) Naj bo $H \subseteq G$ zaprta
        \newline
        \pause
        Pokažimo da za vsak $h \in H$ tudi $U_h \subseteq H$, torej za $g \in G$ velja $ghg^{-1} \in H$ %kar je ravno definicija edinke
        \newline
        \pause
        Recimo, da $\exists h \in H: U_h \nsubseteq H$
        \newline
        \pause
        $x \in U_h \Rightarrow U_x = U_h$, torej $U_x$ seka $H$ in $G - H$.
        \newline
        \pause
        Torej $x$ robna in ni v $H$
        \pause
        $\Rightarrow\!\Leftarrow$

    \end{dokaz}


\end{frame}



\begin{frame}
\frametitle{Uporaba topologije $\mathcal{T}(G)$}

\begin{izrek}
    Če je $\phi: G \rightarrow \Gamma$ homomorfizem grup, potem je praslika $\phi^{-1}(U_\gamma)$ odprta v $\mathcal{T}(G)$ za vsak $\gamma \in \Gamma$.
    \pause
\end{izrek}


\begin{dokaz}
    Zveznost preverimo na bazi: $\forall \gamma \in \Gamma: \phi^{-1}(U_\gamma) \in \mathcal{T}(G)$
    \newline
    \pause
    Definiramo:
    \[ S_\gamma = \bigcup_{\{g | \phi(g) \in U_\gamma\}} U_g \]
    \pause
    $t \in \phi^{-1}(U_\gamma) \Rightarrow \phi(t) \in U_\gamma \Rightarrow t \in U_t \subseteq S_\gamma$
    \newline
    \pause
    Ker to velja za vsak $t$ je {\color{red} $\phi^{-1}(U_\gamma) \subseteq S_\gamma$}
    
\end{dokaz}
\end{frame}

\begin{frame}
\frametitle{Uporaba topologije $\mathcal{T}(G)$}

\begin{dokaz}
    \[ S_\gamma = \bigcup_{\{g | \phi(g) \in U_\gamma\}} U_g \]
    \pause
    Naj bo zdaj $t \in S_\gamma$
    \newline
    \pause
    Obstajata $g,h \in G$, da $\phi(g) \in U_\gamma$ in $t = hgh^{-1}$
    \pause
    \[\phi(t) = \phi(h)\phi(g)\phi(h^{-1}) = \phi(h)\phi(g)\phi(h)^{-1}\]
    \pause
    $\phi(t)$ konjugiran $\phi(g)$, sta v enakem razredu
    \newline
    \pause
    $\phi(g) \in U_\gamma \Rightarrow \phi(t) \in U_\gamma \Rightarrow \phi(t) \in U_\gamma \Rightarrow t \in \phi^{-1}(U_\gamma)$
    \newline
    \pause
    {\color{red} $S_\gamma \subseteq \phi^{-1}(U_\gamma)$}

\end{dokaz}
\end{frame}




%%%%%%%%%%%%%%%%%%%%%%%%%%%%%%%%%%%%%%%%%%%%%%%%%%%%%%%%%%%%%%%%%%%%%%%%%%%%%%%%

\end{document}