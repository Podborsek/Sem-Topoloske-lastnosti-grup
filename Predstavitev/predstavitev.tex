\documentclass{beamer}

\usepackage[slovene]{babel}
\usepackage{amsfonts,amssymb}
\usepackage[utf8]{inputenc}
\usepackage{lmodern}
\usepackage[T1]{fontenc}

\usetheme{Warsaw}

\def\N{\mathbb{N}} % mnozica naravnih stevil
\def\Z{\mathbb{Z}} % mnozica celih stevil
\def\Q{\mathbb{Q}} % mnozica racionalnih stevil
\def\R{\mathbb{R}} % mnozica realnih stevil
\def\C{\mathbb{C}} % mnozica kompleksnih stevil


\def\qed{$\hfill\Box$}   % konec dokaza
\newtheorem{izrek}{Izrek}
\newtheorem{trditev}{Trditev}
\newtheorem{posledica}{Posledica}
\newtheorem{lema}{Lema}
\newtheorem{definicija}{Definicija}
\newtheorem{pripomba}{Pripomba}
\newtheorem{primer}{Primer}
\newtheorem{zgled}{Zgled}
\newtheorem{zgledi}{Zgledi uporabe}
\newtheorem{zglediaf}{Zgledi aritmetičnih funkcij}
\newtheorem{oznaka}{Oznaka}

\title{Topološke lastnosti grup}
\author{Gašper Rotar}
\institute{Fakulteta za matematiko in fiziko}




\begin{document}


%%%%%%%%%%%%%%%%%%%%%%%%%%%%%%%%%%%%%%%%%%%%%%%%%%%%%%%%%%%%%%%%%%%%%

\begin{frame}
\titlepage
\end{frame}

%%%%%%%%%%%%%%%%%%%%%%%%%%%%%%%%%%%%%%%%%%%%%%%%%%%%%%%%%%%%%%%%%%%%%

\begin{frame}

\begin{oznaka}
\[
\N = \{1,2,3,\ldots\}
\]
%\vspace{0.1pt}
\end{oznaka}

\pause
\medskip
\begin{definicija}
\alert{Aritmetična funkcija} je preslikava oblike
\[
f: \N \to A, \quad A \subseteq \C.
\]
\pause
Aritmetična funkcija $f$ je \alert{multiplikativna}, če za poljubni tuji števili $a, b \in \N$ velja:
\[
f(ab) = f(a)f(b).
\]
%\vspace{0.1pt}
\end{definicija}

\end{frame}

\begin{frame}
    \begin{itemize}
        \item refleksivnost: Naj bo $e$ enota $G$, potem $x = exe^{-1}$
        \item simetričnost: $y = gxg^{-1} \Rightarrow x = g^{-1}yg$
        \item tranzitivnost:
    \end{itemize} 
\end{frame}



\end{document}