\documentclass[a4paper,12pt]{article}

\usepackage[slovene]{babel}
\usepackage{amsfonts,amssymb,amsmath}
\usepackage[utf8]{inputenc}
\usepackage[T1]{fontenc}
\usepackage{lmodern}
\usepackage{graphicx}


\def\N{\mathbb{N}} % mnozica naravnih stevil
\def\Z{\mathbb{Z}} % mnozica celih stevil
\def\Q{\mathbb{Q}} % mnozica racionalnih stevil
\def\R{\mathbb{R}} % mnozica realnih stevil
\def\C{\mathbb{C}} % mnozica kompleksnih stevil
\newcommand{\geslo}[2]{\noindent\textbf{#1} {#2}}

\def\qed{$\hfill\Box$}   % konec dokaza
\def\qedm{\qquad\Box}   % konec dokaza v matematičnem načinu
\newtheorem{izrek}{Izrek}
\newtheorem{trditev}{Trditev}
\newtheorem{posledica}{Posledica}
\newtheorem{lema}{Lema}
\newtheorem{pripomba}{Pripomba}
\newtheorem{definicija}{Definicija}
\newtheorem{zgled}{Zgled}

\title{Topološke lastnosti grup \\ 
\Large Seminar}
\author{Gašper Rotar \\
Fakulteta za matematiko in fiziko}

\begin{document}
\maketitle

\section{Uvod}

V tej seminarski bomo obravnavali uporabe topoloških pristopov za študij lastnosti grup.



\section{Dva preprosta izreka}

\begin{definicija}
    Naj bo $N$ podgrupa grupe $G$. $N$ je podgrupa \emph{edinka} grupe $G$, označimo $N \triangleleft G$, če za vse $a \in G$ in $n \in N$ velja $ana^{-1} \in N$.
\end{definicija}

\begin{trditev}
    Za podgrupo $N$ grupe $G$ so naslednji pogoji ekvivalenti:
    \begin{itemize}
    \item[\rm (i)] $N$ je ednika.
    \item[\rm (ii)] $aN \subseteq Na$ za vsak $a \in G$.
    \item[\rm (iii)] $aN = Na$ za vsak $a \in G$.
    \item[\rm (iv)] $aNa^{-1} = N$ za vsak $a \in G$. 
    \end{itemize}
\end{trditev}

\begin{trditev}
    Naj bo $G$ grupa in $\Delta(G) = \{(g,g) \mid g \in G \} \subseteq G \times G$.
    Grupa $G$ je komutativna natanko takrat, ko je $\Delta(G)$ podgrupa edinka grupe $G$, $\Delta(G) \triangleleft G$.
\end{trditev}


\noindent
{\em Dokaz:\/}
    \begin{itemize}
        \item[($\Rightarrow$)] Ker je $G$ komutativna je seveda tudi $G \times G$ komutativna. Torej je vsaka njena podgrupa edinka.
            Sedaj je trebna le še pokazati, da je $\Delta(G) \le G \times G$, kar je enostavno. Naj bosta $a = (\alpha, \alpha), b = (\beta, \beta) \in \Delta(G)$, potem:
            \begin{gather*}
                a \cdot b = (\alpha, \alpha) \cdot (\beta, \beta) = (\alpha \cdot \beta, \alpha \cdot \beta) \in \Delta(G), \\
                (\alpha, \alpha) \cdot (\alpha^{-1}, \alpha^{-1}) = (\alpha \cdot \alpha^{-1}, \alpha \cdot \alpha^{-1}) = (1,1).
            \end{gather*}
            Vidimo, da je $\Delta(G)$ zaprta za opreacijo, inverz $(\alpha, \alpha)$ pa je $(\alpha^{-1}, \alpha^{-1}) \in \Delta(G)$, torej zaprta tudi za invertiranje.
        \item[($\Leftarrow$)] Naj bosta $\alpha, \beta$ elementa grupe $G$ z enoto $1$.
        Potem so $(\alpha,1), (\alpha^{-1}, 1), (\beta, \beta) \in G \times G$ in
            $(\alpha,1)^{-1} = (\alpha^{-1},1)$ ter $(\beta, \beta) \in \Delta(G)$. Potem lahko izračunamo:
            \[(\alpha,1) \cdot (\beta, \beta) \cdot (\alpha^{-1}, 1) = (\alpha\beta\alpha^{-1}, \beta).\]
            Ker je $\Delta(G)$ ednika je zaprta za konjugiranje, torej je $\alpha\beta\alpha^{-1} = \beta \Rightarrow \alpha\beta = \beta\alpha$
    \end{itemize}
\qed

\begin{definicija}
    Topološki prostor $X$ je \emph{Hausdorffov}, če za vsaki različni točki $x_1, x_2 \in X$ obstajata odprti okoloici $U_1$ in $U_2$ za točki $x_1$ in $x_2$, da $U_1 \cap U_2 = \emptyset$.
\end{definicija}

\begin{trditev}
    Naslednje izjave so ekvivalentne:
    \begin{itemize}
        \item[\rm (i)] Prostor $X$ je Hausdorffov.
        \item[\rm (ii)] Za poljuben $x \in X$ je $\bigcap_{U \in \mathcal{U}} \overline{U}$, kjer je $\mathcal{U}$ družina vseh okolic $x$.
        \item[\rm (iii)] Diagonala $\Delta(X) = \{(x,x) \mid x \in X \} $ je zaprt podprostor produkta $X \times X$
    \end{itemize}
\end{trditev}








\section{Konjugiranostna topologija}



\subsection{Konjugiranostni razerdi}

\begin{definicija}
    Element $y$ grupe $G$ je \emph{konjugiran} elementu $x$ iz $G$, če obstaja tak $g \in G$, da je $y = gxg^{-1}$.
\end{definicija}

\begin{trditev}
    Konjugiranost je ekvivalenčna relacija.
\end{trditev}

\noindent
{\em Dokaz:\/}
    Relacija je ekvivalenčna, če je refleksivna, simetrična in tranzitivna. Naj bo $e$ enota grupe $G$ in $x$ poljuben element $G$, potem $x = exe^{-1}$.
    To pomeni, da je poljuben element $G$ konjugiran samemu sebi, relacija je refleksivna.
    Naj bo $y$ konjugiran $x$, torej obstaja $g \in G$, da $y = gxg^{-1}$. Z enostavnim preoblikovanjem enačbe dobimo $x = g^{-1}yg$.
    Torej je tudi $x$ konjugiran $y$, relacija je simetrična.
    Končno pokažimo še tranzitivnost. Če je $y = gxg^{-1}$ in $z = hyh^{-1}$ potem vstavimo $y$ v drugo enačbo da dobimo $z = hgxg^{-1}h^{-1}$.
    Opazimo, da je $(hg)^{-1} = g^{-1}h^{-1}$ torej je $z$ konjugiran $x$.
\qed

\begin{pripomba}
    Ekvivalenčnim razredom za konjugiranost pravimo \emph{konjugiranostni razredi}.
\end{pripomba}

\begin{zgled}
    \begin{itemize}
        \item Če je $G$ komutativna potem za vsak $x,g \in G$ velja $gxg^{-1} = xgg^{-1} = x$, torej je vsak element sam v svojem konjugiranostnem razredu.
        \item kvaternioni
        \item podobne matrike
    \end{itemize}
\end{zgled}


\begin{izrek}
    Naj bo $G$ grupa in označimo $U_h = \{ghg^{-1} \mid g \in G \}$, to je konjugiranostni razred elementa $h$.
    Množica $\Theta = \bigcup_{h \in G}\{U_h\}$ je baza topologije na $G$.
\end{izrek}

\begin{pripomba}
    To topologijo bomo imenovali \emph{konjugiranostna topologija} in jo označili z $\mathcal{T}(G)$.
\end{pripomba}


\noindent
{\em Dokaz:\/}
    Najprej moramo pokazati, da $\Theta$ pokritje cel $G$, kar je enostavno, preprosto vzamemo $\bigcup_{h \in G}U_h$.
    Ta množica je unija odprtih in vsebuje vsak $g \in G$, saj je ta gotovo v $U_g$.

\qed





\subsection{Uporaba}



\begin{izrek}
    $G$ je Abelova natanko takrat, ko je $\mathcal{T}(G)$ Hausdorffova.
\end{izrek}

\noindent
{\em Dokaz:\/}
    \begin{itemize}
        \item[($\Rightarrow$)] $G$ je Abelova natanko takrat, ko $gh = hg $,za vsak $h,g \in G$, ekvivalentno $ghg^{-1} = h$, za vsak $g,h \in G$, kar pa ravno pomeni $U_h = \{ h\}$ za vsak $h \in G$.
        To pomeni, da so v $\mathcal{T}(G)$ enojci odprti, torej je to diskretna topologija. Diskretna topologija je seveda Hausdorffova.
        \item[($\Leftarrow$)] Privzemimo sedaj, da je $\mathcal{T}(G)$ Hausdorffov. Imejmo poljuben $x \in G$ in recimo, da obstaja $y \in U_x$ različen od $x$.
        Ker je $G$ Hausdorffova obstajata odprti okoloici $U_h$ in $U_k$, da $x \in U_h, y \in U_k$ in $U_h \cap U_k = \emptyset$.
        Toda to pomeni, da $x \in U_x \cap U_h$ in $y \in U_x \cap U_k$ torej $U_h = U_x = U_h$, kar pa je v protislovju z $U_h \cap U_k = \emptyset$.
        To pomeni, da $U_x = \{x\}$, kar pa je ravno ekvivalentno komutativnosti $G$.
    \end{itemize} 
\qed


\begin{izrek}
    $H \triangleleft G$, če in samo če je $H$ zaprt v $\mathcal{T}(G)$.
\end{izrek}

\noindent
{\em Dokaz:\/}
    \begin{itemize}
        \item[($\Rightarrow$)] Naj bo $H \triangleleft G$. Dovolj je pokazati, da je $G - H$ odprta, saj iz tega sledi, da je $H$ zaprta.
        Naj bo $x \in G - H$ in $U_x$ konjugiranostni razred $x$, po definiciji je to tudi odprta okolica $x$.
        Recimo, da $U_x \cap H \neq \emptyset$, potem obstajata $g \in G$ in $h \in H$, da $h = gxg^{-1}$.
        Iz tega sledi $x = g^{-1}hg$, ker pa je $H$ edinka po definiciji sledi $x \in H$. Prispeli smo do protislovja, torej $U_x \cap H = \emptyset$.
        Tako smo našli odprto okolico za $x$, ki je vsa v množica $G - H$, torej je ta odprta.
        \item[($\Leftarrow$)] Še pride
    \end{itemize} 
\qed

\begin{izrek}
    Če je $\phi: G \rightarrow \Gamma$ homomorfizem grup, potem je praslika $\phi^{-1}(U_\gamma)$ odprta v $\mathcal{T}(G)$ za vsak $\gamma \in \Gamma$.
\end{izrek}

\noindent
{\em Dokaz:\/}
    Še pride
\qed

\section{Pogeld v druge algebraične strukture}

Zaenkrat nič.


\section*{Angleško-slovenski slovar strokovnih izrazov}

\geslo{normal subgroup}{podgrupa edinka}

\begin{thebibliography}{1}
    \bibitem{Bre}
    M.~Brešar, \emph{Uvod v algebro}, DMFA -- založništvo, Ljubljana, 2018.
\end{thebibliography}
    



\end{document}