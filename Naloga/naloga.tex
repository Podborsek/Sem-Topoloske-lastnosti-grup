\documentclass[a4paper,12pt]{article}

\usepackage[slovene]{babel}
\usepackage{amsfonts,amssymb,amsmath}
\usepackage[utf8]{inputenc}
\usepackage[T1]{fontenc}
\usepackage{lmodern}
\usepackage{graphicx}


\def\N{\mathbb{N}} % mnozica naravnih stevil
\def\Z{\mathbb{Z}} % mnozica celih stevil
\def\Q{\mathbb{Q}} % mnozica racionalnih stevil
\def\R{\mathbb{R}} % mnozica realnih stevil
\def\C{\mathbb{C}} % mnozica kompleksnih stevil
\newcommand{\geslo}[2]{\noindent\textbf{#1} \quad \hangindent=1cm #2\\[{-1}pc]}

\def\qed{$\hfill\Box$}   % konec dokaza
\def\qedm{\qquad\Box}   % konec dokaza v matematičnem načinu
\newtheorem{izrek}{Izrek}
\newtheorem{trditev}{Trditev}
\newtheorem{posledica}{Posledica}
\newtheorem{lema}{Lema}
\newtheorem{pripomba}{Pripomba}
\newtheorem{definicija}{Definicija}
\newtheorem{zgled}{Zgled}

\title{Topološke lastnosti grup \\ 
\Large Seminar}
\author{Gašper Rotar \\
Fakulteta za matematiko in fiziko}

\begin{document}
\maketitle

\section{Uvod}

V tej seminarski bomo obravnavali uporabe topoloških pristopov za študij lastnosti grup.



\section{Dva preprosta izreka}

\begin{definicija}
    Naj bo $N$ podgrupa grupe $G$. $N$ je podgrupa edinka grupe $G$, označimo $N \triangleleft G$, če za vse $a \in G$ in $n \in N$ velja $ana^{-1} \in N$.
\end{definicija}

\begin{trditev}
    Za podgrupo $N$ grupe $G$ so naslednji pogoji ekvivalenti:
    \begin{itemize}
    \item[\rm (i)] $N$ je ednika.
    \item[\rm (ii)] $aN \subseteq Na$ za vsak $a \in G$.
    \item[\rm (iii)] $aN = Na$ za vsak $a \in G$.
    \item[\rm (iv)] $aNa^{-1} = N$ za vsak $a \in G$. 
    \end{itemize}
\end{trditev}

\begin{trditev}
    Naj bo $G$ grupa in $\Delta(G) = \{(g,g) \mid g \in G \} \subseteq G \times G$.
    Grupa $G$ je komutativna natanko takrat, ko je $\Delta(G)$ podgrupa edinka grupe $G$, $\Delta(G) \triangleleft G$.
\end{trditev}


\noindent
{\em Dokaz:\/}
    \begin{itemize}
        \item[($\Rightarrow$)] Ker je $G$ komutativna je seveda tudi $G \times G$ komutativna. Torej je vsaka njena podgrupa edinka.
            Zdaj je trebna le še pokazati, da je $\Delta(G) \le G \times G$, kar je enostavno. Naj bosta $a = (\alpha, \alpha), b = (\beta, \beta) \in \Delta(G)$, potem:
            \begin{gather*}
                a + b = (\alpha, \alpha) + (\beta, \beta) = (\alpha + \beta, \alpha + \beta) \in \Delta(G), \\
                (\alpha, \alpha) + (\alpha^{-1}, \alpha^{-1}) = (\alpha \cdot \alpha^{-1}, \alpha \cdot \alpha^{-1}) = (1,1).
            \end{gather*}
            Vidimo, da je $\Delta(G)$ zaprta za opreacijo, inverz $(\alpha, \alpha)$ pa je $(\alpha^{-1}, \alpha^{-1}) \in \Delta(G)$, torej zaprta tudi za invertiranje.
        \item[($\Leftarrow$)] Naj bosta $\alpha, \beta$ elementa grupe $G$ z enoto $1$.
        Potem so $(\alpha,1), (\alpha^{-1}, 1), (\beta, \beta) \in G \times G$ in
            $(\alpha,1)^{-1} = (\alpha^{-1},1)$ ter $(\beta, \beta) \in \Delta(G)$. Potem lahko izračunamo:
            \[(\alpha,1) \cdot (\beta, \beta) \cdot (\alpha^{-1}, 1) = (\alpha\beta\alpha^{-1}, \beta).\]
            Ker je $\Delta(G)$ ednika je zaprta za invertiranje, torej je $\alpha\beta\alpha^{-1} = \beta \Rightarrow \alpha\beta = \beta\alpha$
    \end{itemize}
\qed

\begin{definicija}
    Topološki prostor $X$ je Hausdorffov, če za vsaki različni točki $x_1, x_2 \in X$ obstajata odprti okoloici $U_1$ in $U_2$ za točki $x_1$ in $x_2$, da $U_1 \cap U_2 = \emptyset$.
\end{definicija}

\begin{trditev}
    Naslednje izjave so ekvivalentne:
    \begin{itemize}
        \item[\rm (i)] Prostor $X$ je Hausdorffov.
        \item[\rm (ii)] Za poljuben $x \in X$ je $\cap_{U \in \mathcal{U}} \overline{U}$, kjer je $\mathcal{U}$ družina vseh okolic $x$.
        \item[\rm (iii)] Diagonala $\Delta(X) = \{(x,x) \mid x \in X \} $ je zaprt podprostor produkta $X \times X$
    \end{itemize}
\end{trditev}


\section{Konjugiranostna topologija}

\begin{definicija}
    Naj bo $G$ grup in označimo $U_h = \{ghg^{-1} \mid g \in G \}$
\end{definicija}

\begin{izrek}
    $G$ je Abelova natanko takrat, ko je $\mathcal{T}(G)$ Hausdorffova.
\end{izrek}

\end{document}