\documentclass[a4paper,12pt]{article}

\usepackage[slovene]{babel}
\usepackage{amsfonts,amssymb,amsmath}
\usepackage[utf8]{inputenc}
\usepackage[T1]{fontenc}
\usepackage{lmodern}
\usepackage{graphicx}


\def\N{\mathbb{N}} % mnozica naravnih stevil
\def\Z{\mathbb{Z}} % mnozica celih stevil
\def\Q{\mathbb{Q}} % mnozica racionalnih stevil
\def\R{\mathbb{R}} % mnozica realnih stevil
\def\C{\mathbb{C}} % mnozica kompleksnih stevil
\newcommand{\geslo}[2]{\noindent\textbf{#1} {#2}}

\def\qed{$\hfill\Box$}   % konec dokaza
\def\qedm{\qquad\Box}   % konec dokaza v matematičnem načinu
\newtheorem{izrek}{Izrek}
\newtheorem{trditev}{Trditev}
\newtheorem{posledica}{Posledica}
\newtheorem{lema}{Lema}
\newtheorem{pripomba}{Pripomba}
\newtheorem{definicija}{Definicija}
\newtheorem{zgled}{Zgled}

\title{Topološke lastnosti grup \\ 
\Large Seminar}
\author{Gašper Rotar \\
Fakulteta za matematiko in fiziko}

\begin{document}
\maketitle

\section{Uvod}

V tej seminarski nalogi bomo obravnavali uporabe topoloških pristopov za študij lastnosti grup.
To bomo storili tako, da bomo prek konjugiranja na naraven način na vsaki grupi definirali topologijo.



\section{Dve preprosti trditvi}

Za motivacijo si najprej poglejmo dve preprosti trditvi, eno iz teorije grup, drugo iz splošne topologije.
Podobnost med tema dvema trditvama nam bo dala idejo za topologijo, ki jo bomo definirali v naslednjem poglavju.
Spomnimo se zdaj definicije podgrupe edinke, ki jo bomo uporabili pri dokazu prve trditve.

\begin{definicija}
    Naj bo $N$ podgrupa grupe $G$. $N$ je podgrupa \emph{edinka} grupe $G$, označimo $N \triangleleft G$, če za vse $a \in G$ in $n \in N$ velja $ana^{-1} \in N$.
\end{definicija}


Prva trditev, ki si jo bomo ogledali je karakterizacija komutativnosti s podgrupo edinko. Pri tem si bomo pomagali z diagonalo $G$.

\begin{trditev}
    Naj bo $G$ grupa in $\Delta(G) = \{(g,g) \mid g \in G \} \subseteq G \times G$ njena diagonala.
    Grupa $G$ je komutativna natanko takrat, ko je $\Delta(G)$ podgrupa edinka grupe $G$, $\Delta(G) \triangleleft G$.
\end{trditev}


\noindent
{\em Dokaz:\/}
    \begin{itemize}
        \item[($\Rightarrow$)] Ker je $G$ komutativna je seveda tudi $G \times G$ komutativna. Torej je vsaka njena podgrupa edinka.
            Sedaj je trebna le še pokazati, da je $\Delta(G) \le G \times G$, kar je enostavno. Naj bosta $a = (\alpha, \alpha), b = (\beta, \beta) \in \Delta(G)$, potem:
            \begin{gather*}
                a \cdot b = (\alpha, \alpha) \cdot (\beta, \beta) = (\alpha \cdot \beta, \alpha \cdot \beta) \in \Delta(G), \\
                (\alpha, \alpha) \cdot (\alpha^{-1}, \alpha^{-1}) = (\alpha \cdot \alpha^{-1}, \alpha \cdot \alpha^{-1}) = (1,1).
            \end{gather*}
            Vidimo, da je $\Delta(G)$ zaprta za množenje. Dalje velja, da je inverz $(\alpha, \alpha)$ ravno $(\alpha^{-1}, \alpha^{-1}) \in \Delta(G)$, torej zaprta tudi za invertiranje.
        \item[($\Leftarrow$)] Naj bosta $\alpha, \beta$ elementa grupe $G$ z enoto $1$.
        Potem so $(\alpha,1), (\alpha^{-1}, 1), (\beta, \beta) \in G \times G$ kjer
            $(\alpha,1)^{-1} = (\alpha^{-1},1)$ ter $(\beta, \beta) \in \Delta(G)$. Potem lahko izračunamo:
            \[(\alpha,1) \cdot (\beta, \beta) \cdot (\alpha^{-1}, 1) = (\alpha\beta\alpha^{-1}, \beta).\]
            Ker je $\Delta(G)$ edinka je zaprta za konjugiranje, kar pomeni $(\alpha\beta\alpha^{-1}, \beta) \in \Delta$,
            torej je $\alpha\beta\alpha^{-1} = \beta$ oziroma $\alpha\beta = \beta\alpha$. Ker to velja za poljubna $\alpha, \beta$ je $G$ komutativna.
    \end{itemize}
\qed

Sedaj, ko smo dokazali prvo trditev, si poglejmo še drugo. Najprej bomo definirali Hausdorffovo lastnost, potem pa jo bomo, kot zgoraj komutativnost, karakterizirali s pomočjo diagonale. 

\begin{definicija}
    Topološki prostor $X$ je \emph{Hausdorffov}, če za vsaki različni točki $x_1, x_2 \in X$ obstajata odprti okolici $U_1$ in $U_2$ za točki $x_1$ in $x_2$, da $U_1 \cap U_2 = \emptyset$.
\end{definicija}

Sledeča trditev poveže Hausdorffovo lastnost z diagonalo prostora.
V naslednji trditvi sta glavni prva in tretja točka, druga je tam predvsem za lažje dokazovaje.

\begin{trditev}
    Naslednje izjave so ekvivalentne:
    \begin{itemize}
        \item[\rm (i)] Prostor $X$ je Hausdorffov.
        \item[\rm (ii)] Za poljuben $x \in X$ je $\bigcap_{U \in \mathcal{U}} \overline{U} = \{x\}$, kjer je $\mathcal{U}$ družina vseh okolic $x$.
        \item[\rm (iii)] Diagonala $\Delta(X) = \{(x,x) \mid x \in X \} $ je zaprt podprostor produkta $X \times X$
    \end{itemize}
\end{trditev}

\noindent
{\em Dokaz:\/}
    Privzemimo, da je prostor Hausdorffov in vzemimo $y \neq x$. Tedaj obstajata okolici $U$ za $x$ in $V$ za $y$, ki sta disjunktni. Torej $y \notin \overline{U}$.
    Tako tudi ni v preseku zaprtij vseh okolic točke $x$. Torej je $\bigcap_{U \in \mathcal{U}} \overline{U} = \{x\}$.
    Sedaj pokažimo, da je $\Delta^C$ odprta v $X \times X$, kar je ekvivalentno temu, da je $\Delta$ zaprta.
    Naj bo $(x,y) \in \Delta^C$. Vemo, da obstaja $U$, ki je odprta okolica $x$, katere zaprtje ne vsebuje $y$.
    Potem je $U \times \overline{U}^C$ odprta okolica $(x,y)$, ki ne seka diagonale.
    Privzemino zdaj da je diagonala zaprta. Če sta točki $x$ in $y$ različni, potem za $(x,y)$ obstaja škatlasta odprta okolica $U \times V$, ki ne seka diagonale.
    To pomeni, da sta $U$ in $V$ odprti in disjunktni, kar pa je ravno zahteva za Hausdorffovo lastnost. \qed



Zdaj, ko smo dokazali obe trditvi, lahko opazimo zanimive paralele med komutativnostjo in Hausdorffovo lastnostjo. Obe lastnosti sta dokaj osnovni,
če namreč pomislimo na najbolj preproste zglede grup nam verjetno na pamet padej najprej $\Z, \Q, \R$, ki so vse komutativne.
Ravno tako se topološke prostore včasih motivira kot posplošitev metričnih prostorov, ki so seveda Hausdorffovi.
Obe se da karakterizirati z diagonalo.
Vprašamo se, ali se da ti dve lastnosti kako konkretno povezati. Odgovor je da, kakor bomo videli v naslednjem poglavju.





\section{Konjugiranostna topologija}

Do povezave med komutativnostjo in Hausdorffovo lastnostjo bomo prišli prek konjugiranosti.
V prvem podrazdelku bomo obnovili definicijo konjugiranosti in nekatere njene lastnosti. Potem bomo na naraven način definirali topologijo na grup.
V drugem podrazdelku pa bomo videli kako postanejo preko prej definirane topologije mnogi topološki in algebraični pojmi ekvivalenti.



\subsection{Konjugiranostni razredi}

Konjugiranost je temeljna relacija na grupi. Elementi, ki so si konjugirani so si močno podobni in se, kar se tiče strukture grupe, skoraj enako obnašajo.

\begin{definicija}
    Element $y$ grupe $G$ je \emph{konjugiran} elementu $x$ iz $G$, če obstaja tak $g \in G$, da je $y = gxg^{-1}$.
\end{definicija}

\begin{trditev}
    Konjugiranost je ekvivalenčna relacija.
\end{trditev}

\noindent
{\em Dokaz:\/}
    Relacija je ekvivalenčna, če je refleksivna, simetrična in tranzitivna. Naj bo $e$ enota grupe $G$ in $x$ poljuben element $G$, potem $x = exe^{-1}$.
    To pomeni, da je poljuben element $G$ konjugiran samemu sebi, relacija je refleksivna.
    Naj bo $y$ konjugiran $x$, torej obstaja $g \in G$, da $y = gxg^{-1}$. Z enostavnim preoblikovanjem enačbe dobimo $x = g^{-1}yg$.
    Torej je tudi $x$ konjugiran $y$, relacija je simetrična.
    Končno pokažimo še tranzitivnost. Če je $y = gxg^{-1}$ in $z = hyh^{-1}$ potem vstavimo $y$ v drugo enačbo da dobimo $z = hgxg^{-1}h^{-1}$.
    Opazimo, da je $(hg)^{-1} = g^{-1}h^{-1}$, torej je $z$ konjugiran $x$.
\qed

\begin{pripomba}
    Ekvivalenčnim razredom za konjugiranost pravimo \emph{konjugiranostni razredi}.
\end{pripomba}

Za lažjo predstavo si poglejmo nekaj zgledov:

\begin{zgled}
        Če je $G$ komutativna potem za vsak $x,g \in G$ velja $gxg^{-1} = xgg^{-1} = x$, torej je vsak element sam v svojem konjugiranostnem razredu.
        Še več, če je vsak element v svojem razredu potem za vsak $g,h \in G$ velja $h = ghg^{-1}$, ekvivalentno $hg = gh$.
        Torej je grupa komutativna natanko takrat, ko je vsak element v svojem konjugiranostnem razredu.
\end{zgled}

\begin{zgled}
    Kvaternionska grupa $Q = \{\pm 1, \pm i, \pm j, \pm k\}$ ni komutativna. Z enostavnim a zamudnim računanjem pokažemo,
    da so njeni konjugiranostni razredi $\{1\}, \{-1\}, \{\pm i\}, \{\pm j\}, \{\pm k\}$. 
\end{zgled}

\begin{zgled}
    Pravimo, da sta si dve matriki $A$ in $B$ podobni, če obstaja taka obrnljiva matrika $P$, da je $B = P^{-1}AP$.
    Torej je v $GL_n(\R)$ podobnost ravno konjugiranost. Kot vemo iz linearne algebre predstavljajo podobne matrike isto linearno preslikavo,
    vendar glede na različne baze.
\end{zgled}

Opremljeni z razumevanjem konjugiranostnih razredov bomo lahko dokazali prvi pomemben izrek, na katerem bodo temeljili vsi nadaljni.

\begin{izrek}
    Naj bo $G$ grupa in označimo $U_h = \{ghg^{-1} \mid g \in G \}$, to je konjugiranostni razred elementa $h$.
    Množica $\Theta = \bigcup_{h \in G}\{U_h\}$ je baza topologije na $G$.
\end{izrek}

\begin{pripomba}
    Topologijo, ki jo kot baza podaja $\Theta$ bomo imenovali \emph{konjugiranostna topologija} in jo označili z $\mathcal{T}(G)$.
\end{pripomba}


\noindent
{\em Dokaz:\/}
    Najprej moramo pokazati, da $\Theta$ pokritje cel $G$, kar je enostavno, preprosto vzamemo $\bigcup_{h \in G}U_h$.
    Ta množica je unija odprtih in vsebuje vsak $g \in G$, saj je ta gotovo v $U_g$.
    Naj bosta zdaj  $U_h$ in $U_k$ bazni množici. Pokazati moramo, da za vsak $x \in U_h \cap U_k$ obstaja bazna okolica $x$, ki je pod $U_h \cap U_k$.
    Toda, če se konjugiranostna razreda sekata v eni točki potem sta enaka in tudi konjugiranostni razred te toče je enak, torej je $U_x = U_h = U_k$.
    Torej je $x \in U_x \subseteq U_h \cap U_k$. Pokazali smo, da je $\Theta$ res baza neke topologije.
\qed

Tako smo na naraven način na vsaki grupi definirali topologijo.
Kakšna ta topologija je in kako je povezana z lastnostmi grupe $G$ bomo videli v naslednjem razdelku.
Dokazali bomo tesno povezavo med lastnostmi grupe $G$ in topologije $\mathcal{T}(G)$.




\subsection{Uporaba topologije $\mathcal{T}(G)$}

Najprej lahko pokažemo, da sta si trditvi iz prvega razdelka tako podobni zato, ker sta komutativnost in Hausdorffov lastnost za $\mathcal{T}(G)$ tako rekoč enakovredni.

\begin{izrek}
    $G$ je Abelova natanko takrat, ko je $\mathcal{T}(G)$ Hausdorffova.
\end{izrek}

\noindent
{\em Dokaz:\/}
    \begin{itemize}
        \item[($\Rightarrow$)] $G$ je Abelova natanko takrat, ko $gh = hg $, za vsak $h,g \in G$, ekvivalentno $ghg^{-1} = h$, za vsak $g,h \in G$, kar pa ravno pomeni $U_h = \{ h\}$ za vsak $h \in G$.
        To pomeni, da so v $\mathcal{T}(G)$ enojci odprti, torej je to diskretna topologija. Diskretna topologija je seveda Hausdorffova.
        \item[($\Leftarrow$)] Privzemimo sedaj, da je $\mathcal{T}(G)$ Hausdorffov. Imejmo poljuben $x \in G$ in recimo, da obstaja $y \in U_x$ različen od $x$.
        Ker je $G$ Hausdorffova obstajata odprti okolici $U_h$ in $U_k$, da $x \in U_h, y \in U_k$ in $U_h \cap U_k = \emptyset$.
        Toda to pomeni, da $x \in U_x \cap U_h$ in $y \in U_x \cap U_k$ torej $U_h = U_x = U_h$, kar pa je v protislovju z $U_h \cap U_k = \emptyset$.
        To pomeni, da $U_x = \{x\}$, kar pa je ravno ekvivalentno komutativnosti $G$.
    \end{itemize} 
\qed

Trditvi, ki smo ju uporabili za motivacijo sta komutativnost ter Hausdorffovo lastnost karakterizirali prek diagonale.
V enem primeru je morala biti edinka, v drugem pa zaprta. Naslednji izrek pove, da sta tudi te dve lastnosti v $\mathcal{T}(G)$ enakovredni.

\begin{izrek}
    $H \triangleleft G$, če in samo če je $H$ zaprt v $\mathcal{T}(G)$.
\end{izrek}

\noindent
{\em Dokaz:\/}
    \begin{itemize}
        \item[($\Rightarrow$)] Naj bo $H \triangleleft G$. Dovolj je pokazati, da je $G - H$ odprta, saj iz tega sledi, da je $H$ zaprta.
        Naj bo $x \in G - H$ in $U_x$ konjugiranostni razred $x$, po definiciji je to tudi odprta okolica $x$.
        Recimo, da $U_x \cap H \neq \emptyset$, potem obstajata $g \in G$ in $h \in H$, da $h = gxg^{-1}$.
        Iz tega sledi $x = g^{-1}hg$, ker pa je $H$ edinka po definiciji sledi $x \in H$. Prispeli smo do protislovja, torej $U_x \cap H = \emptyset$.
        Tako smo našli odprto okolico za $x$, ki je vsa v množici $G - H$, torej je ta odprta.
        \item[($\Leftarrow$)] Naj bo $H \subseteq G$ zaprta, oziroma $G - H$ odprta. Pokazali bomo, da je za vsak $h \in H$ tudi
        $U_h \subseteq H$, to pomeni, da za vsak $g \in G$ velja $ghg^{-1} \in H$, kar je ravno definicija edinke. Recimo, da bi obstajal tak $h \in H$,
        da $U_h \nsubseteq H$. To je mogoče le, če obstaja $x$, ji je v $U_h$ in hkrati ni v $H$. Toda iz $x \in U_h$ sledi $U_x = U_h$, torej $U_x$ seka tako $H$ kot $G - H$.
        Očitno je, da vsaka odprta okolica $x$ vsebuje celoten $U_x$, kar pomeni, da vsaka okolica točke $x$ seka tako $H$ kot $G - H$. Drugače povedano, $x$ je mejna točka množice $H$.
        Ker je $H$ zaprta, mora vsebovati vse svoje mejne točke, torej tak $x$ ne obstaja.
    \end{itemize} 
\qed

Pokažimo še, da je vsak homomorfizem grup glede na konjugiranostno topologijo zvezna preslikava.

\begin{izrek}
    Če je $\phi: G \rightarrow \Gamma$ homomorfizem grup, potem je praslika $\phi^{-1}(U_\gamma)$ odprta v $\mathcal{T}(G)$ za vsak $\gamma \in \Gamma$.
\end{izrek}

\noindent
{\em Dokaz:\/}
    Zveznost je dovolj preverjati na bazi, torej je treba pokazati, da je za vsak $\gamma \in \Gamma$ praslika $\phi^{-1}(U_\gamma)$ odprta v $G$.
    V ta namen definiramo
    \[ S_\gamma = \bigcup_{\{g | \phi(g) \in U_\gamma\}} U_g .\]
    $S_\gamma$ je unija odprtih množic torej je odprta. Z obojestransko inkluzijo pokažimo, da je $S_\gamma = \phi^{-1}(U_\gamma)$.
    Če je $t \in \phi^{-1}(U_\gamma)$, potem je $\phi(t) \in U_\gamma$, torej je $t \in U_t \subseteq S_\gamma$. Ker to velja za vsak $t$ je $\phi^{-1}(U_\gamma) \subseteq S_\gamma$.
    Naj bo zdaj $t \in S_\gamma$. Potem obstajata taka $g,h \in G$, da je $\phi(g) \in U_\gamma$ in $t = hgh^{-1}$. Zdaj lahko uporabimo $\phi$ na $t$, da dobimo
    \[\phi(t) = \phi(h)\phi(g)\phi(h^{-1}) = \phi(h)\phi(g)\phi(h)^{-1} ,\] kjer smo pri drugem enačaju upoštevali, da je $\phi$ homomorfizem.
    Torej je $\phi(t)$ konjugiran z $\phi(g)$, kar pomeni, da sta v enakem konjugiranostnem razredu. Ker pa je $\phi(g) \in U_\gamma$, je potem tudi $\phi(t) \in U_\gamma$,
    kar pomeni $\phi(t) \in U_\gamma$, dalje $t \in \phi^{-1}(U_\gamma)$. Tako imamo še drugo inkluzijo, ker to velja za vsak $t \in S_\gamma$.

\qed

Enostavno lahko premislimo, da implikacija v drugo smer ne velja. Vsaka konstantna preslikava je namreč zvezna. Če ne slika ravno v enoto, potem se seveda enota ne preslika v enoto,
torej ne more biti homomorfizem.

\section{Pogled v druge algebraične strukture}

Vprašamo se, ali lahko kaj podobnega definiramo tudi v kolobarjih? Analog podgrupe edinke v kolobarju je ideal, to je taka podgrupa za seštevanje $I$ v kolobarju $K$,
da za vsak $i \in I$ in $k \in K$ velja $ik \in I$ in $ki \in I$. Hitro pa ugotovimo, da za ideale ne velja podobna trditev o diagonali kot za grupe. Če bi bila namreč diagonala
ideal bi za poljubne $a,b,d \in K$ veljalo $(a,b)(d,d) \in \Delta(K)$, kar bi pomenilo $ad = bd$ oziroma $(a-b)d = 0$. To bi torej veljalo tudi pri $a = 1$ in $b = 0$,
kar bi pomenilo, da je $d$ vedno $0$. To pa je z izjemo trivialnega kolobarja nemogoče.
Na kolobarjih torej preprosta definicija kakšne podobne topologije ni mogoča. Ali morda vseeno obstaja kakšna bolj zapletena posplošitev pa je izven obsega te naloge.


\section*{Angleško-slovenski slovar strokovnih izrazov}

\geslo{conjugate}{konjugiran}

\geslo{conjugacy class}{konjugiranostni razred}

\geslo{ideal}{ideal}

\geslo{normal subgroup}{podgrupa edinka}

\geslo{ring}{kolobar}


\begin{thebibliography}{1}
    \bibitem{Bre}
    M.~Brešar, \emph{Uvod v algebro}, DMFA -- založništvo, Ljubljana, 2018.
    \bibitem{Pav}
    P.~Pavešič, \emph{Splošna topologija}, DMFA -- založništvo, Ljubljana, 2017.
    \bibitem{Kohl}
    T. Kohl, When Abelian = Hausdorff, \emph{The College Mathematics Journal}, 43, 3, 213-215
\end{thebibliography}
    



\end{document}